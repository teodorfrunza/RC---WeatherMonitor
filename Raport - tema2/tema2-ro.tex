\documentclass{llncs}
%
\usepackage{tikz-uml}
%
\usepackage{listings}
\usepackage{color}
\definecolor{dkgreen}{rgb}{0,0.6,0}
\definecolor{gray}{rgb}{0.5,0.5,0.5}
\definecolor{mauve}{rgb}{0.58,0,0.82}
%
\lstset{frame=tb,
  language=C++,
  aboveskip=3mm,
  belowskip=3mm,
  showstringspaces=false,
  columns=flexible,
  basicstyle={\small\ttfamily},
  numbers=none,
  numberstyle=\tiny\color{gray},
  keywordstyle=\color{blue},
  commentstyle=\color{dkgreen},
  stringstyle=\color{mauve},
  breaklines=true,
  breakatwhitespace=true,
  tabsize=3
}
%
\usepackage[T1]{fontenc}
\usepackage[utf8]{inputenc}
\usepackage[romanian]{babel}
\usepackage{combelow}
\useshorthands{'}
\defineshorthand{'s}{\cb{s}}
\defineshorthand{'t}{\cb{t}}
\defineshorthand{'S}{\cb{S}}
\defineshorthand{'T}{\cb{T}}
\begin{document}
%
\title{Weather Monitor}
%
\subtitle{Grad Dificultate C}
%
\author{Frunză Teodor-Octavian}
%
\institute{Universitatea "Alexandru Ioan Cuza"
\and  Facultatea de Informatică \\
\email{frunza.teodor@info.uaic.ro}}
%
\maketitle
%
\begin{abstract}
Acest document reprezintă un raport amănun'tit asupra proiectului "Weather Monitor", fiind alcătuit atât din elemente arhitecturale cât 'si din elemente reprezentate de interac'tiunea utilizatorului cu aplica'tia.
\end{abstract}
%
\section{Introducere}
%
\subsection{Enun'tul Problemei}
Să se scrie o aplica'tie client/server pentru managementul (e.g., listare, modificare, 'stergere) informa'tiilor meteo pentru o anumită zonă. La un port separat se va oferi posibilitatea actualizării informa'tiilor meteo privitoare la o localitate sau mul'time de localită'ti ale zonei considerate.
%
\subsection{Analiza Problemei}
Din enun'tul problemei deducem că ni se cere crearea unei aplica'tii alcătuite dintr-un server concurent 'si un client ce se poate conecta la acesta.
Astfel, pentru a rezolva această problemă vom căuta spre a utiliza o metodă optimă de transmitere a datelor de la useri multipli la server 'si invers fără a exista riscul de a pierde bi'ti de date sau de a corupe mesajele trimise. Mai mult, vom încerca optimizarea acesteia pentru a înlătura posibilită'tile de supraîncărcare a re'telei sau timpii excesivi de lungi de a'steptare a răspunsurilor.
%
\section{Tehnologii Uitilzate}
%
\subsection{TCP/IP}
Vom utiliza TCP/IP ca protocol de comunicare în re'teaua aplica'tiei "Weather Monitor" datorită aplica'tiilor sale vaste în problemele de tip conexiune biunivocă client / server. De asemenea, vom creea un server ce utilizează protocolul de tip TCP/IP sub formă concurentă pentru a permite utilizarea simultană a serverului de către mai mul'ti clien'ti fără a corupe sau pierde datele individuale. În ceea ce urmează voi prezenta câteva caracteristici ale protocolului de transmisie TCP/IP datorită cărora a fost ales pentru acest proiect 'si motivul excluderii protocolului UDP.
%
\subsection{Caracteristici TCP/IP}
%
\begin{enumerate}
%
\item Oferă servicii orientate-conexiune, full duplex;
\item Conexiunile sunt sigure pentru transportul fluxurilor de octe'ti;
\item Vizeaza oferirea calită'tii maxime a serviciilor;
\item Controlează fluxul de date (stream-oriented).
\end{enumerate}
%
\subsection{De ce nu folosim UDP?}
%
\begin{enumerate}
\item Oferă servicii minimale de transport, având riscul de a pierde informa'tii;
\item Nu oferă controlul fluxului de date;
\item Nu este orientat conexiune.
\end{enumerate}
%
Astfel se observă că utilizarea protocolului UDP pentru trimiterea informa'tiei între client 'si server ar fi deficitar în cazul nostru, având un risc destul de mare de a pierde informa'tii pe parcurs.
%
\section{Arhitectura Aplica'tiei}
%
\subsection{Metoda de Func'tionare}
%
Aplica'tia "Weather Monitor" va fi alcătuită din următoarele entită'ti/participan'ti:
%
\begin{enumerate}
\item Baza de date / Fi'sier(e) de stocare a informa'tiilor meteo;
\item Server TCP/IP concurent pe baza de fork;
\item Client.
\end{enumerate}
%
Clientul va avea patru op'tiuni din care va putea alege:
%
\begin{enumerate}
\item Afi'sarea datelor meteo actuale (afisareVreme());
\item Alegerea altui ora's (schimbaOras());
\item Actualizarea datele meteo (actualizareVreme());
\item 'Stergerea unui ora's (stergeOras()).
\end{enumerate}
%
\subsection{Diagrama UML}
%
\begin{tikzpicture}
\umlclass[x=8]{Server}{
	+  port : int \\ 
	+  ip : long \\
}{
	+  searchByUserId(userId : int) : void \\
	+  afisareVreme(oras : string) : void \\
	+  schimbaOras(oras : string) : void \\
	+  actualizareVreme() : void \\
	+  stergeOras(oras : string) : void
}
\umlclass[x=0,type=abstract]{AbstractClient}{
	+ alegeOptiune() : int
}{}
\umlclass[y=-5]{Client}{
	+ userId : int
}{}
\umlassoc[mult1=1, mult2=1..*, pos1=0.1, pos2=0.9, name=has]{Server}{AbstractClient}
\umlinherit{Client}{AbstractClient}
\end{tikzpicture}
%
\section{Detalii de Implementare}
%
\subsection{'Scenarii de utilizare}
%
\begin{tikzpicture}
\umlactor[y=-3]{Client}
\umlactor[x=12,y=-3]{Server}
\umlusecase[x=3,y=-3,name=case0]{Connect server}
\umlusecase[x=8,y=0,name=case1]{Afiseaza meteo}
\umlusecase[x=8,y=-2,name=case2]{Schimba Oras}
\umlusecase[x=8,y=-4,name=case3]{Actualizeaza vreme}
\umlusecase[x=8,y=-6,name=case4]{Sterge Oras}
\umlusecase[x=8,y=-8,name=info]{Send Info}
\umlusecase[x=12,y=0,name=db]{Acces DB}
\umlinclude{case0}{case1}
\umlinclude{case0}{case2}
\umlinclude{case0}{case3}
\umlinclude{case0}{case4}
\umlassoc{Client}{case0}
\umlassoc{case1}{Server}
\umlassoc{case2}{Server}
\umlassoc{case3}{Server}
\umlassoc{case4}{Server}
\umlassoc{Server}{db}
\umlassoc{Client}{info}
\umlassoc{info}{Server}
\end{tikzpicture}
%
\section{Cod Relevant}
%
\subsection{Client}
%
\begin{lstlisting}
count = 0;
if (connect (sd, (struct sockaddr *) &server,sizeof (struct sockaddr)) == -1){
	 perror ("[client]Eroare la connect()");
	 return errno;
}

if(count == 0){
	if (write (sd, userId, 5) <= 0){
     		perror ("[client]Eroare la write() spre server\n");
		return errno;
    	}
	if (read (sd, msg, 100) < 0){
      		perror ("[client]Eroare la read() de la server.\n");
      		return errno;
    	}
  printf ("[client]Mesajul primit este:", msg);
  count++;
}
else{
	count ++;
	bzero (msg, 2);
	printf ("[client]1) Afiseaza meteo \n");
	printf ("[client]2) Schimba oras \n");
	printf ("[client]3)Actualizeaza vreme \n");
	printf ("[client]3)Sterge oras \n");
	printf ("[client]0) Inchide conexiunea \n")
	printf ("[client]Introduceti optiunea dorita: \n");
	fflush (stdout);
	read (0, msg, 2);
	while(msg != 1 || msg !=2 || msg != 3 || msg != 4) {
		printf ("[client]Ati introdus o optiune inexistenta. Va rog introduceti optiunea dorita din cele existente(1,2,3,4: ");
		read (0, msg, 2);
	}
}
\end{lstlisting}
%
Am ales la client această secven'tă deoarece reprezintă realizarea conexiunii cu serverul 'si trimiterea automată a id-ului clientului către server.
În func'tie de acest Id i se va afi'sa vremea personalizată de ultimele setări făcute. De exemplu, dacă clientul avea ora'sul Ia'si setat ini'tial,
se va afisa vremea din acea loca'tie. De asemenea, se poate observa 'si un prim prototip de meniu 'si o verificare
pentru alegerea op'tiunii. Dacă clientul alege o op'tiune inexistentă, acesta va trebui să reintroducă un caracter până când acesta reprezintă o op'tiune validă.
%
\subsection{Server}
%
\begin{lstlisting}
client = accept (sd, (struct sockaddr *) &from, &length);
if(fork()==0){
	close(sd);
}
if (client < 0){
	  perror ("[server]Eroare la accept().\n");
	  continue;
}
if (read (client, msg, 100) <= 0){
	  perror ("[server]Eroare la read() de la client.\n");
	  close (client);	
	  continue;		
}
if (verfYId(msg) == true){
	bzero(msgrasp,100);
	importWeather(msgrasp,msg);
	if (write (client, msgrasp, 100) <= 0){
	 	perror ("[server]Eroare la write() catre client.\n");
	  	continue;		
	}
}
if(verifyExit(msg) == true){
	close(client);
}
else{
	bzero (msg, 100);
	if (read (client, msg, 100) <= 0){
	  	perror ("[server]Eroare la read() de la client.\n");
	 	 close (client);	
	  	continue;		
	}
}
//verifyId reprezinta o functie care verifica daca datele primite de la client sunt  de forma Id. Daca sunt de aceasta forma aducem vremea prestabilita din fisierul acelui user
//importWeather -> aduce vremea personalizata pentru client
//verifExit -> daca este 0 inchide consexiunea
\end{lstlisting}
%
Am ales la server această secven'tă deoarece reprezintă acceptarea clientului, crearea copilului 'si închiderea conexiunii în procesul părinte 'si de asemenea
primirea ini'tială a id-ului clientului. Pe baza acestuia se importă din baza de date / fi'sier(e) ultimele setări 'si se afi'sează vremea. După acceptarea id-ului, verificarea existen'tei
acestuia 'si afi'sarea datelor meteorologice predefinite de utlizator, acesta a'steaptă o op'tiune de la client. Dacă este 0 serverul închide conexiunea cu clientul.
%
\section{Concluzii}
%
Până în momentul de fa'tă avem realizate atât arhitectura aplica'tiei (UML) cât 'si 'scenariile de utilizare a aplica'tiei (USE CASE). Mai mult, avem 'si un prototip de client 'si server, urmând să completăm cu func'tiile de verificare 'si parsare a textului. De îmbunătă'tit ar fi securitatea, viteza 'si rezolvarea buggurilor din aplica'tie.
%
\section{Bibliografie}
%
În realizarea acestui document s-au utilizat următoarele resurse:
%
\begin{enumerate}
\item https://profs.info.uaic.ro/~computernetworks/cursullaboratorul.php
\item https://profs.info.uaic.ro/~computernetworks/files/NetEx/S5/servTcpIt.c
\item https://profs.info.uaic.ro/~computernetworks/files/NetEx/S5/cliTcpIt.c
\item https://profs.info.uaic.ro/~computernetworks/ProiecteNet2017.php
\item https://stackoverflow.com/questions/3175105/writing-code-in-latex-document
\item https://tex.stackexchange.com/questions/354089/how-to-give-a-name-to-an-association-with-tikz-uml
\item tikz-uml-userguide.pdf
\item llncsdoc.pdf
\end{enumerate}
\end{document}